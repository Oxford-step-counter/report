\part{Conclusion}
\chapter{Conclusion}

This project set out to achieve two primary goals: develop a general step counter implemented on a smartphone that uses an accelerometer as its data source and develop a sleep detection algorithm that uses a wrist mounted accelerometer as its data source. 

The generalized smartphone step counter exhibits very high accuracies with a median of 96.8\% across a variety of scenarios. A number of scenarios like phone in pocket, phone in hand, etc. were examined and specific parameters for these scenarios were derived with even higher levels of accuracy than for the generic parameters derived for all scenarios. The general algorithm was further validated against three data recordings from patients which resulted in similar statistics demonstrating its generalised capabilities. The step-counter algorithm could also be optimised as it was developed modularly and has an infrastructure built around it to enable rapid iteration. The database collected in the optimization and validation stage will be released publicly to provide a resource for future research.

The sleep detection algorithm was developed using an existing database. It outperformed previous algorithms evaluated on this database with respect to the accuracy of metrics like the extraction of Total Asleep Time (10\% reduction in median error). Compared to previous attempts, which tended to overestimate sleep, this algorithm was more balanced in its estimation of the states (awake and asleep) which is a contributing factor to the higher accuracy. As with the step counting algorithm, the sleep detection algorithm is also built with future development in mind, as minimal changes would be needed to introduce new features into the feature vector for classification. 

These two algorithms introduce new tools into the medical healthcare professional's toolkit that enable rapid and inexpensive data collection. This data can then be used to help inform care decisions. The data collection can be done at home, without the presence of medical professionals or the use of expensive equipment. For patients suffering from conditions such as chronic obstructive pulmonary disease or heart failure that require constant monitoring and evaluation, this can both simplify the patients' lives and reduce the cost to the healthcare system. The proliferation of smartphones and smart devices mean that this sort of approach can become more widespread, increasing the data available to healthcare professionals, and driving down the costs of healthcare provision.