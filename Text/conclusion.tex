\part{Conclusion}
\chapter{Conclusion}

This project set to do two main things: develop a general step counter implemented on a smartphone that uses an accelerometer as its data source and develop a sleep detection algorithm that uses a wrist mounted accelerometer as its data source. 

The generalized smartphone step counter exhibits very high accuracies with a median of 96.8\% across a variety of scenarios. A number of scenarios like: phone in pocket, phone in hand, etc. were examined and specific parameters for these scenarios were derived with even higher levels of accuracy than the general parameters. The general algorithm was further validated against three data recordings from patients which resulted in similar statistics showing that it is truly general. The step counter algorithm is also modular and can be subject to further improvement and optimization as it was developed modularly and has an infrastructure built around it to enable rapid iteration. The database collected in the optimization and validation stage will be released publicly to provide a resource for future resources.

The sleep detection algorithm was developed against an existing database and outperformed previous algorithm attempts on this database with 10\% error reduction in Total Asleep Time. Compared to previous attempts, which tended to overestimate sleep, this algorithm managed to be more balanced in its estimation which is a contributing factor to the higher accuracy. Similar to the step counting algorithm, the sleep detection algorithm is also built with iterations in mind, as minimal changes would be needed to introduce new features into the feature vector for classification. 

These two algorithms introduce new tools into the medical healthcare professional's toolkit that enable rapid and cheap data collection. This data can then be used to help inform medical decisions. The data collection can be done at home, without the presence of medical professionals or expensive equipment. For patients suffering from conditions such as chronic obstructive pulmonary disease and heart failure that require constant monitoring and evaluation, this can simplify both the patients' lives and reduce the cost, of both time and money, to the healthcare system. The expanding proliferation of smartphones and smart devices mean that this sort of approach can become more widespread, reducing healthcare costs and increasing the data available to professionals. 